\documentclass[aspectratio=1610,hyperref={colorlinks,linkcolor=}]{beamer}

\usepackage{fontspec}
\usepackage{microtype}

\usepackage{booktabs}
\usepackage{color}
\usepackage{fancyvrb}
\usepackage[l2tabu,orthodox]{nag}
\usepackage{upquote}

\usecolortheme{beaver}

\usefonttheme[onlymath]{serif}
\useinnertheme{rounded}
\setbeamertemplate{itemize items}[circle]
\setbeamertemplate{navigation symbols}{}
\setbeamertemplate{footline}[frame number]
\setbeamertemplate{blocks}[rounded][shadow=true]
\setbeamertemplate{section in toc}[square]
\setbeamertemplate{section page}
{
    \begin{center}
        \usebeamerfont{section title}\usebeamercolor[fg]{frametitle}\Huge\insertsection\par
    \end{center}
}
\setbeamercolor{block title}{bg=frametitle.bg}
\setbeamercolor{block body}{bg=frametitle.bg}
\setbeamercolor{block title alerted}{bg=frametitle.bg}
\setbeamercolor{block body alerted}{bg=frametitle.bg}

\title{Best \LaTeX\ practices}
\author{Jan Gosmann}
\date{February 17, 2017}

\AtBeginSection{
    \begin{frame}
        \sectionpage{}
    \end{frame}
}

\begin{document}
\maketitle

\begin{frame}
    \frametitle{Why is this hard?}
    \begin{itemize}
        \item One of the oldest pieces of software you are using.
        \item \TeX\ was released 1978, feature complete since 1989.
        \item \LaTeX\ was released 1985.
        \item Best practices change over time.
        \item Many recommendations on the Internet are outdated.
    \end{itemize}
\end{frame}


\begin{frame}
    \frametitle{Outline}
    \tableofcontents
\end{frame}

\section{The \TeX\ environment}
\begin{frame}
    \frametitle{Why and why not to use \LaTeX?}
    \begin{block}{Pro}
        \begin{itemize}
            \item Text based format good for version control and collaboration.
            \item Beautiful typography.
            \item Semantic markup (focus on content).
        \end{itemize}
    \end{block}
    \begin{block}{Contra}
        \begin{itemize}
            \item Horrible language.
            \item Compile errors hard to track down.
            \item Sometimes hard to get exactly what you want.
        \end{itemize}
    \end{block}
\end{frame}

\begin{frame}
    \frametitle{\TeX\ flavours}
    \begin{itemize}
        \item \TeX\@: Original ``low-level'' typesetting system by Donald Knuth.
        \item \LaTeX\@: Abstraction level build on \TeX\ to isolate the user from typesetting decisions.
        \item ConTeXt\@: In some ways similar to \LaTeX, but provides easy access to advanced typographic control. Uses ``unified system'' instead of individual packages.
    \end{itemize}
    Probably best to stick to \LaTeX\ because of abstraction level and templates provided by publishers.
\end{frame}

\begin{frame}
    \frametitle{\LaTeX\ compilers}
    \begin{itemize}
        \item \texttt{latex}: Used to be the original \LaTeX\ compiler to DVI\@, but now defaults to \texttt{pdflatex}.
        \item \texttt{pdflatex}: Compiles to PDF and supports special PDF features.
        \item \texttt{xelatex}: Adds support for PDF, UTF-8, and system fonts.
        \item \texttt{lualatex}: Adds support for PDF, UTF-8, system fonts, and Lua scripting.
    \end{itemize}
    Recommendations: Use \texttt{lualatex} if you can (first stable version was released last year), \texttt{xelatex} is also a good choice. Sometimes you have to resort to \texttt{pdflatex} (e.g., for the CogSci template or specific \texttt{microtype} features).  Use \texttt{latex} only if you absolutely require specific packages (e.g., pstricks) not supported by the other compilers.
\end{frame}

\begin{frame}
    \frametitle{How to install \LaTeX?}
    \begin{itemize}
        \item Via the package manager of your Linux distribution.
        \item I prefer \href{https://www.tug.org/texlive/}{TeX Live} available for all major OS\@.  Yearly releases, gives you \emph{all} the packages in the most recent versions.
    \end{itemize}
\end{frame}

\begin{frame}[fragile]
    \frametitle{Useful tools}
    \begin{itemize}
        \item \texttt{texdoc}: Quickest way to pull up package documentations.
        \item \texttt{chktex}: Static checker, warns you about things easy to overlook. I recommend using it as an automatic checker in your favourite editor (e.g., with the Syntastic plugin in Vim).
        \item \texttt{latexmk}: Best way to compile \LaTeX.
    \end{itemize}
    \begin{beamerboxesrounded}{My .latexmkrc}  % chktex 26
        \begin{verbatim}
$lualatex = 'lualatex -synctex=1 %O %S';
$pdflatex = 'pdflatex -synctex=1 %O %S';
$xelatex = 'xelatex -synctex=1 %O %S';
$pdf_mode = 4;  # 1 for pdflatex, 4 for lualatex, 5 for xelatex
        \end{verbatim}
    \end{beamerboxesrounded}
\end{frame}

\begin{frame}[fragile]
    \frametitle{Preamble}
    \begin{beamerboxesrounded}{pdflatex}
        \begin{verbatim}
\usepackage[utf8]{inputenc}
\usepackage[T1]{fontenc}
        \end{verbatim}
    \end{beamerboxesrounded}
    \begin{beamerboxesrounded}{xelatex and lualatex}
        \begin{verbatim}
\usepackage{fontspec}
        \end{verbatim}
    \end{beamerboxesrounded}
    Also, consider loading the \texttt{microtype} package, but you might want to disable protrusion for the table of contents (see section 9 of the manual).
\end{frame}

\section{Markup}
\begin{frame}[fragile]
    \frametitle{Semantic markup}
    \begin{itemize}
        \item Mark what things are, not what they should look like.
        \item Allows you to easily adjust formatting.
    \end{itemize}
    \begin{beamerboxesrounded}{Example: Common definitions I use}
        \begin{verbatim}
\newcommand{\mat}[1]{\bm{#1}}  % matrix
\newcommand{\vc}[1]{\bm{#1}}  % vector
        \end{verbatim}
    \end{beamerboxesrounded}
\end{frame}

\begin{frame}[fragile]
    \frametitle{Special text styles}
    \begin{itemize}
        \item Use \verb+\emph+ to emphasize text because it uses italics correction and can be nested: \verb+This is \emph{emphasized with \emph{even more} emphasis}.+ produces ``This is \emph{emphasized with \emph{even more} emphasis}.''.
        \item Consider the semantic difference between \verb+\emph{apples}, \emph{oranges}, and \emph{bananas}+ (three individual items) and \verb+\emph{A, B, and C}+ (a single entity like a proper name).
        \item Use \verb+\text??+ commands for formatting instead of \verb+{\?? ...}+, e.g.~\verb+\textbf{text}+ instead of \verb+{\bf text}+. The \verb+\text??+ commands can be nested and \verb+\textit+ has italics correction.
    \end{itemize}
\end{frame}

\begin{frame}[fragile]
    \frametitle{Equations}
    \begin{itemize}
        \item Use \verb+equation+ and amsmath environments, not \verb+$$+ for display equations.
        \item Equations are part of sentences.
        \item Do not forget punctuation.
        \item Do not introduce paragraph breaks around equations by additional empty lines (if it does not end the paragraph).
    \end{itemize}
    \begin{beamerboxesrounded}{Example}
        \begin{Verbatim}[commandchars=~\[\]]
The input is given by
\begin{equation}  %[~color[red] no empty line before this]
    J(x) = \alpha_i e_ x(t) [~color[red]\text{.}]
\end{equation}
        \end{Verbatim}
    \end{beamerboxesrounded}
\end{frame}

\begin{frame}[fragile]
    \frametitle{Formatting in equations}
    \begin{itemize}
        \item Only variables (and measured physical constants) should be in italics (default).
        \item Common functions and operators like $\max$, $\cos$ should be made upright by prefixing them with \verb+\+, e.g.~\verb+\max+ ($\max$), \textbf{not} \verb+max+ ($max$). %chktex 35
        \item Use \verb+\DeclareMathOperator+ from the amsmath package to declare new operator names.
        \item Subscripts etc.~that are not a variable should be made upright with \verb+\mathrm+ or \verb+\mathsf+, e.g.~\verb+\tau_{\mathrm{syn}}+ ($\tau_{\mathrm{syn}}$), \textbf{not} \verb+\tau_{syn}+ ($\tau_{syn}$), but \verb+a_i,\ 0 < i < 4+ ($a_i,\ 0 < i < 4$).
        \item Latex treats each letter as a single variable. Enclose multi-letter variables with either \verb+\mathit+ or \verb+\mathsfit+ to ensure proper spacing.
        \item The \verb+commath+ package provides some useful commands and make it easy to correctly set differentials.
    \end{itemize}
\end{frame}

\section{Special characters}
\begin{frame}[fragile]
    \frametitle{Spaces}
    \begin{itemize}
        \item Use non-breaking space \verb+~+ where you would never want a linebreak. For example, \verb+Dr.~Who+, \verb+Figure~\ref{fig:a}+, or \verb+(a)~this, or (b)~that+.
        \item Use either non-breaking space \verb+~+ or explicit normal length space \verb*+\ + after periods that do not end a sentence because the intersentence space in English can be larger than the interword space.
        \item Use intersentence spacing when sentence ends in all caps word. For example, \verb+Independent accumulator is abbreviated with IA\@.+ (Chktex warns about this.)
    \end{itemize}
\end{frame}

\begin{frame}[fragile]
    \frametitle{Quotes}
    \begin{itemize}
        \item Use \verb+``text''+ to produce actual quotes: ``text''.
        \item \textbf{Not} \verb+"text"+ which gives "text". % chktex 18
        \item Same for single quotes.
    \end{itemize}
\end{frame}

\begin{frame}[fragile]
    \frametitle{Horizontals bars aka dashes}
    \begin{tabular}{rccl}
        Hyphen & - & \verb+-+ & connects words, e.g., good-hearted \\ % chktex 8
        en dash & -- & \verb+--+ & span or range of numbers, e.g., chapters 5--9 \\ % chktex 8
        em dash & --- & \verb+---+ & denotes break in sentence or sets off source of a quote \\
        Minus & $-$ & \verb+$-$+ & or \verb+\textminus+ \\
        Plus & $+$ & \verb$+$ & (here for comparison to minus)  \\
    \end{tabular}
    \begin{itemize}
        \item Vertical alignment comparison: {\Huge -\,--\,---\,$-+$}. % chktex 8
        \item Minus has same height and width as the horizontal bar of the plus.
        \item The em dash for sentence breaks is traditionally set without spaces around it.
        \item Sometimes (in newspapers?) it is set with spaces. Consider non-breaking spaces if you use spaces.
    \end{itemize}
\end{frame}

\section{Package recommendations}
\begin{frame}[fragile]
    \frametitle{References}
    \begin{itemize}
        \item Use \verb+biblatex+ package with \verb+biber+ backend if possible (most modern package, most features, UTF-8 support).
        \item Templates provided by journals often use older packages like \verb+natbib+ and require the old \verb+bibtex+ backend.
        \item Use appropriate citation commands, e.g.~\verb+\parencite+ vs \verb+\textcite+ with \verb+biblatex+.
        \item Avoid parenthesis in parenthesis (might require some manual piecing together with different lower level citation commands).
        \item Using reference management software can be a good idea. (I use Zotero with the ``Better Bib(La)TeX'' plugin, others are Papers for macOS and Mendeley.)  % chktex 36
    \end{itemize}
\end{frame}

\begin{frame}[fragile]
    \frametitle{Tables}
    \begin{itemize}
        \item Never, ever use vertical rules.
        \item Never use double rules.
        \item Caption goes above table, not below (in contrast to figures).
        \item Align numeric columns on period.
        \item Use \verb+booktabs+ package and read its documentation.
    \end{itemize}
    \begin{center}
        \begin{tabular}{@{}llr@{}} \toprule
            \multicolumn{2}{c}{Item} \\ \cmidrule(r){1-2}
            Animal & Description & Price (\$)\\ \midrule
            Gnat & per gram & 13.65 \\
            & each & 0.01 \\
            Gnu & stuffed & 92.50 \\
            Emu & stuffed & 33.33 \\
            Armadillo & frozen & 8.99 \\ \bottomrule
        \end{tabular}
    \end{center}
\end{frame}

\begin{frame}[fragile]
    \frametitle{Some further package recommendations}
    \begin{itemize}
        \item \verb+nag+ warns about usage of outdated packages and commands (load with \verb$\usepackage[l2tabu,orthodox]{nag}$, consider adding the \verb+abort+ option).
        \item \verb+amsmath+ provides a number of useful equation environments and related commands.
        \item \verb+graphicx+ (it is more modern and has more functionality than \verb+graphics+)
        \item \verb+siunitx+ provides configurable formatting for numbers and quantities with units.
        \item \verb+tikz+ allows to create great figures, but has a steep learning curve.
        \item \verb+komascript+ is a collection of feature-rich document classes and packages.
    \end{itemize}
\end{frame}

\section{Grab bag}
\begin{frame}
    \frametitle{Collaboration}
    \begin{itemize}
        \item \url{overleaf.com}
            \begin{itemize}
                \item Pros: web interface, Git access
                \item Cons: version tracking is lacking, only \LaTeX\ inline comments
            \end{itemize}
        \item \url{github.com}
            \begin{itemize}
                \item Pros: PRs allow discussion threads in code, good version tracking
                \item Cons: more complicated to use?
            \end{itemize}
        \item \url{sharelatex.com} (have not tried it yet)
        \item Use one sentence per line. (You might want to set your editor to soft wrap mode.)
    \end{itemize}
\end{frame}

\begin{frame}
    \frametitle{Forward and inverse search}
    \begin{itemize}
        \item Jump from tex-file to corresponding spot in the PDF (forward).
        \item Jump from the PDF to the corresponding spot in the tex-file (backward).
        \item Requires to compile tex with \texttt{-synctex=1}.
        \item Further setup depends on editor and PDF viewer.
    \end{itemize}
\end{frame}

\begin{frame}
    \frametitle{Further reading}
    \begin{itemize}
        \item \url{https://www.olivierverdier.com/posts/2013/07/15/modern-latex/}
        \item \url{http://practicaltypography.com/}
        \item \texttt{texdoc l2tabuen}
    \end{itemize}
\end{frame}

% inverse and forward search

\end{document}
