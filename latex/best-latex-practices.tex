\documentclass{beamer}

\usepackage{nag}

\usecolortheme{beaver}
%\defaultfontfeatures{Mapping=tex-text}
%\setsansfont{Helvetica Neue}

\useinnertheme{rounded}
\setbeamertemplate{itemize items}[circle]
\setbeamertemplate{navigation symbols}{}
\setbeamertemplate{footline}[frame number]
\setbeamertemplate{blocks}[rounded][shadow=true]
\setbeamercolor{block title}{bg=frametitle.bg}
\setbeamercolor{block body}{bg=frametitle.bg}
\setbeamercolor{block title alerted}{bg=frametitle.bg}
\setbeamercolor{block body alerted}{bg=frametitle.bg}

\title{Best \LaTeX\ practices}
\author{Jan Gosmann}
\date{February 17, 2017}

\begin{document}
\maketitle

\begin{frame}
    \frametitle{Why is this hard?}
    \begin{itemize}
        \item One of the oldest pieces of software you are using.
        \item \TeX\ was released 1978, feature complete since 1989.
        \item \LaTeX\ was released 1985.
        \item Best practices change over time.
        \item Many recommendations on the Internet are outdated.
    \end{itemize}
\end{frame}

\begin{frame}
    \frametitle{Why and why not to use \LaTeX?}
    \begin{block}{Pro}
        \begin{itemize}
            \item Text based format good for version control and collaboration.
            \item Beautiful typography.
            \item Semantic markup (focus on content).
        \end{itemize}
    \end{block}
    \begin{block}{Contra}
        \begin{itemize}
            \item Horrible language.
            \item Compile errors hard to track down.
            \item Sometimes hard to get exactly what you want.
        \end{itemize}
    \end{block}
\end{frame}

\begin{frame}
    \frametitle{What is \dots?}
    \begin{itemize}
        \item \TeX\@: Original ``low-level'' typesetting system by D.~Knuth.
        \item \LaTeX\@: Abstraction level build on \TeX\ to isolate the user from typesetting decisions.
        \item ConTeXt\@: In some ways similar to \LaTeX, but provides easy access to advanced typographic control. Uses ``unified system'' instead of individual packages.
    \end{itemize}
    Probably best to stick to \LaTeX\ because of abstraction level and templates provided by publishers.
\end{frame}

\begin{frame}
    \frametitle{What is \dots?}
    \begin{itemize}
        \item \texttt{latex}: Original \LaTeX\ compiler to DVI\@.
        \item \texttt{pdflatex}: Compiles to PDF and supports special PDF features.
        \item \texttt{xelatex}: Adds support for PDF, UTF-8, and system fonts.
        \item \texttt{lualatex}: Adds support for PDF, UTF-8, system fonts, and Lua scripting.
    \end{itemize}
    Recommendations: Use \texttt{lualatex} if you can (first stable version was released last year), \texttt{xelatex} is also a good choice. Sometimes you have to resort to \texttt{pdflatex} (e.g., for the CogSci template). Use \texttt{latex} only if you absolutely require specific packages (e.g., pstricks) not supported by the other compilers.
\end{frame}
    
\begin{frame}
    \frametitle{How to install \LaTeX?}
    \begin{itemize}
        \item Via the package manager of your Linux distribution.
        \item I prefer TeX Live available for all major OS\@. Yearly releases, gives you \emph{all} the packages in the most recent versions.
    \end{itemize}
\end{frame}

\begin{frame}[fragile]
    \frametitle{Useful tools}
    \begin{itemize}
        \item \texttt{texdoc}: Quickest way to pull up package documentations.
        \item \texttt{chktex}: Static checker, warns you about things easy to overlook. I recommend using it as an automatic checker in your favourite editor (e.g., with the Syntastic plugin in Vim).
        \item \texttt{latexmk}: Best way to compile \LaTeX.
    \end{itemize}
    \begin{beamerboxesrounded}{My .latexmkrc}
        \begin{verbatim}
$pdflatex = "lualatex -synctex=1 %O %S";
$pdf_mode = 1;
$postscript_mode = $dvi_mode = 0;
        \end{verbatim}
    \end{beamerboxesrounded}
\end{frame}


% inverse and forward search
% one sentence per line
% collab

% standard packages: graphicx, beamer, tikz, nag, siunitx
% komascript

\end{document}
